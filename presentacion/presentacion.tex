\documentclass{beamer}
%\usetheme{Goettingen}
%\usetheme{Singapore}
\usetheme{Darmstadt}
\usecolortheme{seahorse}

\AtBeginSection[]{
	\begin{frame}
		\vfill
		\centering
		\begin{beamercolorbox}[sep=8pt,center,shadow=true,rounded=true]{title}
			\usebeamerfont{title}\insertsectionhead\par%
		\end{beamercolorbox}
		\vfill
	\end{frame}
}

\usepackage[utf8]{inputenc}
\usepackage[spanish, es-tabla]{babel}


% Fracciones más grandes
\newcommand\ddfrac[2]{\frac{\displaystyle #1}{\displaystyle #2}}



%Information to be included in the title page:
\title{Determinación de Órbitas Elípticas}
\subtitle{El Método de Laplace}
\author{Simón López Vico}
\institute{Doble Grado en Matemáticas e Ingeniería Informática\\Universidad de Granada}
\date{Septiembre de 2020}


\begin{document}

\frame{\titlepage}

\begin{frame}{Índice}
\tableofcontents
\end{frame}

\section{Introducción}

\begin{frame}
\begin{figure}[H]
\centering
\includegraphics[scale=0.15]{images/problem_construct_1.png}
\end{figure}
\end{frame}

\begin{frame}
\begin{figure}[H]
\centering
\includegraphics[scale=0.15]{images/problem_construct_2.png}
\end{figure}
\end{frame}

\begin{frame}
\begin{figure}[H]
\centering
\includegraphics[scale=0.15]{images/problem_construct_3.png}
\end{figure}
\end{frame}


\section{El Método de Laplace}

\begin{frame}
\frametitle{Aproximando las derivadas}
Tres observaciones: $(\lambda_1,\mu_1,\nu_1)$, $(\lambda_2,\mu_2,\nu_2)$, $(\lambda_3,\mu_3,\nu_3)$.\pause
\vspace{0.66666cm}

\textbf{Diferencia regresiva} ($t_2>t_1$)
\[
\lambda'_{12}=\frac{\lambda_2-\lambda_1}{t_2-t_1}
\]

\textbf{Diferencia progresiva} ($t_2<t_3$)
\[
\lambda'_{23}=\frac{\lambda_3-\lambda_2}{t_3-t_2}
\]

\textbf{Diferencia centrada} ($t_2-t_1=t_3-t_2$)
\[
\lambda'_{2}=\frac{\lambda'_{12}+\lambda'_{23}}{2}
\]

\end{frame}

\begin{frame}
\begin{figure}[H]
\centering
\includegraphics[scale=0.2]{images/triangle_tfg.png}
\end{figure}
\end{frame}

\begin{frame}
Soluciones para $\phi$:
\[
\sin^4{\phi}=M\sin{(\phi + m)}
\]
\vspace{0.6666cm}
\pause

Condiciones: \hspace{0.5cm}
$
\left\{
\begin{array}{l}
\phi\in(0,\pi)\\
\phi<\pi-\psi
\end{array}
\right.
$
\end{frame}

\begin{frame}{Valores para $\phi$}
\begin{figure}[H]
\centering
\includegraphics[scale=0.1]{images/phi_solution_m_negative_M_near_1.png}
\end{figure}

\end{frame}

\begin{frame}{Utilizamos los valores calculados}
\[\rho'=\ddfrac{D_2}{D}\left(\ddfrac{1}{R^3}-\ddfrac{1}{r^3}\right)\]\\
\vspace{1cm}

\pause
\begin{columns}
\column[]{0.35\textwidth}
\textbf{Posición:}\\
\vspace{0.4cm}
$
\left\{
\begin{array}{l}
	x=\rho\lambda-X\\
	y=\rho\mu-Y\\
	z=\rho\nu-Z
\end{array}
\right.
$
\column[]{0.35\textwidth}
\textbf{Velocidad:}\\
\vspace{0.4cm}
$
\left\{
\begin{array}{l}
	x'=\rho'\lambda+\rho\lambda'-X'\\
	y'=\rho'\mu+\rho\mu'-Y'\\
	z'=\rho'\nu+\rho\nu'-Z'
\end{array}
\right.
$
\end{columns}
\end{frame}

\section{La Órbita Completa}
\begin{frame}{Elementos orbitales}
Utilizando $r=(x,y,z)$ y $v=r'$ podemos obtener $(a,e,i,\omega,\Omega)$.\\
\pause
\vspace{1cm}
\textbf{Elipse:}
\hspace{0.75cm}
$
(a\cos{\theta}+ae, a\sqrt{1-e^2}\sin{\theta}, 0), \; \; \; \; \theta\in(0,2\pi)
$
\end{frame}


\begin{frame}{Posición de la órbita}
% Diapositiva para dibujar

%\begin{figure}[H]
%\centering
%\includegraphics[scale=0.25]{images/omega_i.png}
%\end{figure}
\end{frame}

\section{Bondad del método}

\begin{frame}{Herramientas para el desarrollo}
\begin{itemize}
\item \textbf{Python:} Numpy, Matplotlib, Astropy, etc.
\item \textbf{Tkinter}
\end{itemize}
\end{frame}

\end{document}