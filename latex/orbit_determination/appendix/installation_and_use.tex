\markedchapter{Instalación y uso}{Instalación y uso del programa informático}
\section{Requisitos para la instalación.}
Todo el software implementado se ha desarrollado y probado en Ubuntu 16.04.7 LTS, pero no debería haber problema en ejecutarlo en cualquier otra distribución Linux, en Windows o MacOS. Aún así, la instalación se explicará para sistemas Linux.\\

Dado que todo el desarrollo está hecho en Python, será el primer requisito que debemos de instalar. Para ello, utilizamos la orden:
\begin{lstlisting}[style=Console]
sudo apt install python3
\end{lstlisting}

Por defecto, debería venir instalado el paquete \textit{Tkinter} que necesitaremos para la interfaz gráfica, pero si no fuera así ejecutaremos el comando:
\begin{lstlisting}[style=Console]
sudo apt install python3-tk
\end{lstlisting}

\section{Instalación del programa.}
Para obtener todo el código del proyecto, será necesario descargarlo del repositorio de GitHub \href{https://github.com/smnlpz/OrbitDetermination}{OrbitDetermiantion} u obtenerlo mediante la orden \texttt{git} utilizando el comando:
\begin{lstlisting}[style=Console]
git clone https://github.com/smnlpz/OrbitDetermination.git
\end{lstlisting}

A continuación, debemos de instalar todos los paquetes de Python que se han utilizado para el desarrollo del programa. Para ello utilizaremos \texttt{pip3}, que debería de venir instalado por defecto, pero si no fuera así solo tendríamos que utilizar:
\begin{lstlisting}[style=Console]
sudo apt install python3-pip
\end{lstlisting}

En el repositorio que acabamos de descargar, entramos al directorio \texttt{code/} y corremos la orden:
\begin{lstlisting}[style=Console]
pip3 install -r requirements.py
\end{lstlisting}

Con todo esto, ya deberíamos de tener todo a punto para utilizar el programa.\\

\section{Uso del programa.}
Para ejecutar la interfaz gráfica desarrollada para la determinación de órbitas, simplemente tendremos que entrar al directorio \texttt{code/} y utilizar la orden:
\begin{lstlisting}[style=Console]
python3 orbitTroya.py
\end{lstlisting}

Tras ello, se abrirá la ventana del programa y bastará con introducir tanto los distintos valores de ascensión recta y declinación del cuerpo observado como el momento de observación para obtener una aproximación de su órbita completa.\\

Debemos tener en cuenta que para el correcto funcionamiento del programa es necesario que se esté conectado a internet.

