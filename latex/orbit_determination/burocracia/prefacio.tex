
\thispagestyle{empty}
%\cleardoublepage

%\thispagestyle{empty}




\clearpage{\thispagestyle{empty}\cleardoublepage}

\thispagestyle{empty}
\begin{center}
{
	\large\bfseries Determinación de Órbitas Elípticas: el Método de Laplace
}\\
\end{center}

\begin{center}
	Simón López Vico\\
\end{center}

%\vspace{0.7cm}
\noindent{\textbf{Palabras clave}: elipse, mecánica celeste, determinación de órbitas, Laplace, derivación numérica, Python.}\\

\vspace{0.7cm}
\noindent{\textbf{Resumen}}\\

Poner aquí el resumen.


\clearpage{\thispagestyle{empty}\cleardoublepage}




\thispagestyle{empty}
\begin{center}
{
	\large\bfseries Orbit Determination: Laplace's Method
}\\
\end{center}

\begin{center}
	Simón López Vico\\
\end{center}

%\vspace{0.7cm}
\noindent{\textbf{Keywords}: ellipse, celestial mechanics, orbit determination, Laplace, numerical differentation, Python.}\\

\vspace{0.7cm}
\noindent{\textbf{Abstract}}\\

Write here the abstract in English.

\newpage






\thispagestyle{empty}

\noindent\rule[-1ex]{\textwidth}{2pt}\\[4.5ex]

Yo, \textbf{Simón López Vico}, alumno de la titulación Doble Grado en Matemáticas e Ingeniería Informática de la \textbf{Facultad de Ciencias de la Universidad de Granada} y la \textbf{Escuela Técnica Superior de Ingenierías Informática y de Telecomunicación de la Universidad de Granada}, con DNI 26504148Y, autorizo la ubicación de la siguiente copia de mi Trabajo Fin de Grado en la biblioteca del centro para que pueda ser consultada por las personas que lo deseen.

\vspace{6cm}

\noindent Fdo: Simón López Vico

\vspace{2cm}

\begin{flushright}
Granada a 7 de septiembre de 2020.
\end{flushright}

\clearpage{\thispagestyle{empty}\cleardoublepage}

\thispagestyle{empty}

\noindent\rule[-1ex]{\textwidth}{2pt}\\[4.5ex]

D. \textbf{Rafael Ortega Ríos}, Profesor del Departamento de Matemática Aplicada de la Universidad de Granada.

\vspace{0.5cm}

D. \textbf{Sergio Alonso Burgos}, Profesor del Departamento de Lenguajes y Sistemas Informáticos de la Universidad de Granada.


\vspace{0.5cm}

\textbf{Informan:}

\vspace{0.5cm}

Que el presente trabajo, titulado \textit{\textbf{Determinación de Órbitas Elípticas: el Método de Laplace}}, ha sido realizado bajo su supervisión por \textbf{Simón López Vico}, y autorizamos la defensa de dicho trabajo ante el tribunal que corresponda.

\vspace{0.5cm}

Y para que conste, expiden y firman el presente informe en Granada a 7 de septiembre de 2020.

\vspace{1cm}

\textbf{Los directores:}

\vspace{5cm}

\noindent \textbf{Rafael Ortega Ríos \ \ \ \ \ Sergio Alonso Burgos}


\clearpage{\thispagestyle{empty}\cleardoublepage}

\thispagestyle{empty}


\begin{Huge}
\textbf{Agradecimientos}
\end{Huge}
\thispagestyle{empty}

\vspace{1cm}

Gracias a mis tutores, por toda la ayuda que me han brindado durante todos estos meses de pandemia, a pesar de todas las dificultades del camino. A Rafael Ortega Ríos por descubrirme la mecánica celeste y hacer que me guste aún más el estudio del universo que nos rodea, y a Sergio Alonso Burgos por su atención y ayuda para todo el desarrollo informático.\\

Pero en especial, gracias a mi familia, por soportar todos los momentos de agobio durante estos años de estudio y ayudarme a seguir adelante. La cuesta arriba se hubiera hecho más difícil sin vuestra ayuda, gracias de corazón.

\clearpage{\thispagestyle{empty}\cleardoublepage}

\thispagestyle{empty}

