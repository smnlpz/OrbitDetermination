
\thispagestyle{empty}
%\cleardoublepage

%\thispagestyle{empty}




\clearpage{\thispagestyle{empty}\cleardoublepage}

\thispagestyle{empty}
\begin{center}
{
	\large\bfseries Determinación de Órbitas Elípticas: el Método de Laplace
}\\
\end{center}

\begin{center}
	Simón López Vico\\
\end{center}

%\vspace{0.7cm}
\noindent{\textbf{Palabras clave}: elipse, mecánica celeste, determinación de órbitas, Laplace, derivación numérica, Python.}\\

\vspace{0.7cm}
\noindent{\textbf{Resumen}}\\

El estudio de métodos matemáticos para la determinación de la órbita de un cuerpo celeste con simples observaciones desde la Tierra es un tema que lleva desarrollándose desde que Newton presentara el primer método de determinación en 1687. Desde entonces han ido apareciendo nuevos y mejores procedimientos para obtener mayor precisión en la órbita, además de que la mejora en las tecnologías para la observación ha ayudado mucho a aumentar la eficacia de cada uno de estos métodos. Con todo esto, podemos conocer la posición en cualquier momento de un cuerpo celeste, lo que será importante para futuros viajes espaciales.\\

A lo largo de este trabajo, estudiaremos el método de determinación desarrollado por Laplace en 1780, que se bastará de únicamente tres observaciones de un objeto en la bóveda celeste en diferentes momentos para intentar aproximar de manera óptima su trayectoria alrededor del Sol.\\

Para empezar, introduciremos los principales conceptos de la mecánica celeste, así como la manera de realizar y medir las observaciones; esto será necesario a la hora de entender el método laplaciano. A continuación, veremos el método de Laplace, basado en aproximación numérica y resolución de ecuaciones, y el cálculo para la transformación de posición y velocidad a coordenadas astronómicas, que definirán la órbita del objeto por completo. Tras ello, se realizará un estudio sobre el número de soluciones que se pueden obtener utilizando el método de determinación y los cálculos para discernir entre cuál de todas las soluciones posibles es la válida.\\

Por último, se implementará un software capaz de aproximar una órbita mediante el método laplaciano para facilitar la aplicación de este, pues requiere de muchas operaciones. Por medio del programa desarrollado, se verán diferentes casos de aproximación con distintos cuerpos para así determinar la eficacia del método de Laplace estudiado.





\clearpage{\thispagestyle{empty}\cleardoublepage}




\thispagestyle{empty}
\begin{center}
{
	\large\bfseries Orbit Determination: Laplace's Method
}\\
\end{center}

\begin{center}
	Simón López Vico\\
\end{center}

%\vspace{0.7cm}
\noindent{\textbf{Keywords}: ellipse, celestial mechanics, orbit determination, Laplace, numerical differentation, Python.}\\

\vspace{0.7cm}
\noindent{\textbf{Abstract}}\\

The study of mathematical methods for determining the orbit of an astronomical object with simple observations from the Earth is a subject that has been developing since Newton presented the first method of determination in 1687. Ever since, new and better methods for obtaining greater accuracy in orbit have been developed, and improved technologies for observation have greatly helped to increase the effectiveness of each of these methods. With all this, we can determine the position of a celestial body at any time, which will be important for future space travel.\\

In this work, we will study the orbit determination method developed by Laplace in 1780, which, with only three observations of an object at three different times, will try to approximate optimally its movement around the Sun.\\

First, we will introduce the main concepts of celestial mechanics, as well as how to make and measure the observations; this will be necessary when understanding the Laplacian method. Next, we will see the Laplace method, based on numerical approximation and equation solving, and the calculation for the transformation of position and velocity into astronomical coordinates, which will define the object's orbit completely. After this, a study will be made on the number of solutions that can be obtained using the determination method and the calculations in order to discern between which among all the possible solutions is the valid one.\\

Finally, we will implement a software capable of approximating an orbit by means of the Laplacian method to facilitate its application, as many operations are required. By means of the program developed, different cases of approximation with different objects will be seen in order to determine the effectiveness of the Laplace method studied.

\newpage






\thispagestyle{empty}

\noindent\rule[-1ex]{\textwidth}{2pt}\\[4.5ex]

Yo, \textbf{Simón López Vico}, alumno de la titulación Doble Grado en Matemáticas e Ingeniería Informática de la \textbf{Facultad de Ciencias de la Universidad de Granada} y la \textbf{Escuela Técnica Superior de Ingenierías Informática y de Telecomunicación de la Universidad de Granada}, con DNI 26504148Y, autorizo la ubicación de la siguiente copia de mi Trabajo Fin de Grado en la biblioteca del centro para que pueda ser consultada por las personas que lo deseen.

\vspace{6cm}

\noindent Fdo: Simón López Vico

\vspace{2cm}

\begin{flushright}
Granada a 7 de septiembre de 2020.
\end{flushright}

\clearpage{\thispagestyle{empty}\cleardoublepage}

\thispagestyle{empty}

\noindent\rule[-1ex]{\textwidth}{2pt}\\[4.5ex]

D. \textbf{Rafael Ortega Ríos}, Profesor del Departamento de Matemática Aplicada de la Universidad de Granada.

\vspace{0.5cm}

D. \textbf{Sergio Alonso Burgos}, Profesor del Departamento de Lenguajes y Sistemas Informáticos de la Universidad de Granada.


\vspace{0.5cm}

\textbf{Informan:}

\vspace{0.5cm}

Que el presente trabajo, titulado \textit{\textbf{Determinación de Órbitas Elípticas: el Método de Laplace}}, ha sido realizado bajo su supervisión por \textbf{Simón López Vico}, y autorizamos la defensa de dicho trabajo ante el tribunal que corresponda.

\vspace{0.5cm}

Y para que conste, expiden y firman el presente informe en Granada a 7 de septiembre de 2020.

\vspace{1cm}

\textbf{Los tutores:}

\vspace{5cm}

\noindent \textbf{Rafael Ortega Ríos \ \ \ \ \ Sergio Alonso Burgos}


\clearpage{\thispagestyle{empty}\cleardoublepage}

\thispagestyle{empty}


\begin{Huge}
\textbf{Agradecimientos}
\end{Huge}
\thispagestyle{empty}

\vspace{1cm}

Gracias a mis tutores, por toda la ayuda que me han brindado durante todos estos meses de pandemia, a pesar de todas las dificultades del camino. A Rafael Ortega Ríos por descubrirme la mecánica celeste y hacer que me guste aún más el estudio del universo que nos rodea, y a Sergio Alonso Burgos por su atención y ayuda para todo el desarrollo informático.\\

Pero en especial, gracias a mi familia, por soportar todos los momentos de agobio durante estos años de estudio y ayudarme a seguir adelante. La cuesta arriba se hubiera hecho más difícil sin vuestra ayuda, gracias de corazón.

\clearpage{\thispagestyle{empty}\cleardoublepage}

\thispagestyle{empty}

